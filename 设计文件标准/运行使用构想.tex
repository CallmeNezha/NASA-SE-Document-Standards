\documentclass{ctexart}
\usepackage{graphics}

\title{运行使用构想\\{\large Version 1.0}}
\author{江子健}
\date{\today}

\begin{document}

\maketitle
\newpage

\tableofcontents

\newpage
\section{范围}

\subsection{标识信息}

\subsection{系统用途}

\subsection{文档概述}

\newpage
\section{参考文档}



\newpage
\section{背景信息}

\newpage
\section{现有系统和运行使用}

\newpage
\section{拟议系统操作概述}

\subsection{任务}

\subsection{运行使用政策与约束}

\subsection{人员}

\subsubsection{组织结构}

\subsubsection{人员档案}

\subsection{支持概念与环境}

\subsubsection{变更的依据和内容}

\subsubsection{所需变更的总结}

\subsubsection{考虑过但未包含的变更}

\subsection{影响总结}

\newpage
\section{系统概述}

\subsection{系统范围}

\subsection{系统的目标和目的}

\subsection{用户与操作人员}

\subsection{系统的接口和边界}

\subsection{系统的状态和模式}

\subsection{系统能力}

\subsection{系统架构}

\newpage
\section{运行过程}

\newpage
\section{其他运行需求}

\subsection{任务需求}

\subsection{人员需求}

\subsubsection{人员类型x}

\subsection{质量因素}

\newpage
\section{拟议系统分析}

\subsection{优势概述}

\subsection{劣势概述/局限性总结}

\subsection{考虑过的替代方案与权衡}

\subsection{不同用户类别的影响概述}

\subsection{监管影响}

\subsection{其他影响}

\newpage
\section{附录 A:缩写词、简称与术语表}

\newpage
\section{附录 B:系统运行场景}

\subsection{附录 B.1 运行过程}

\subsubsection{附录 B.1.x 场景x}

\subsection{附录 B.2 常见场景与条件}

\end{document}
